\usepackage[british]{babel}
\usepackage{lipsum}
\usepackage[T1]{fontenc}
\usepackage[utf8]{inputenc}
\usepackage{lmodern}
\usepackage{microtype}
\usepackage{ragged2e}
\usepackage{multicol}
\usepackage{paralist}
\usepackage{ulem}
\usepackage{blindtext}
%Bibliography
\usepackage[backend=biber, style = authoryear, sorting=nyt]{biblatex}
    \bibliography{references.bib}

\renewbibmacro*{cite}{%
  \iffieldundef{shorthand}
    {\ifthenelse{\ifnameundef{labelname}\OR\iffieldundef{labelyear}}
       {\usebibmacro{cite:label}%
        \setunit{\printdelim{nonameyeardelim}}}
       {\printnames{labelname}%
        \setunit{\printdelim{nameyeardelim}}}%
     \usebibmacro{cite:labeldate+extradate}%
     \setunit{\addcomma\space}%
     \usebibmacro{journal}
     \setunit{\addcomma\space}%
     \usebibmacro{title}}
    {\usebibmacro{cite:shorthand}}}

%Graphikeinbindungen
\usepackage{graphicx}
%\usepackage{sidecap}  
\usepackage{pgfplots}
    \pgfplotsset{compat = 1.15}
\usepackage{tikz}
    \usetikzlibrary{calc,quotes,arrows.meta,automata}
    \usetikzlibrary{overlay-beamer-styles}
\usepackage{tikz-dimline}
\usepackage[compat=1.1.0]{tikz-feynman}
%Farbeinstellungen für Hyperlinks und Pythoncode
\usepackage{xcolor}
    \definecolor{myred}  {HTML}{A3061E}
    \definecolor{myblue} {RGB} {0,63,119}
    \definecolor{myyellow} {cmy} {0,0.263,0.741}
    \definecolor{mygreen} {HTML}{0B6E4F}
    \colorlet{myorange} {myyellow!60!myred}
    \colorlet{myviolett} {myred!50!myblue!80}
%für Python code
\usepackage{listings}
	\lstdefinestyle{python}{
	language         = Python                   ,
	basicstyle       = \ttfamily                ,
	keywordstyle     = \color{myred}            ,
	identifierstyle  = \color{myblue}           ,
	stringstyle      = \color{mygreen}          ,
	commentstyle     = \color{black!50}         ,
	numberstyle      = \color{black!50}\tiny    ,
	numbers          = left                     ,
	belowcaptionskip = \baselineskip            ,
	}
%
\usepackage{caption}
    \captionsetup{%
        font=small,%
        format=plain,%
        labelfont=bf,%
        labelsep=colon,%
        margin=10pt,%
        textfont=sl,%
        singlelinecheck=true%
    }
%Gegen Widowline
\usepackage[defaultlines=2,all]{nowidow}
%Mathe-. und Physikdinge
\usepackage{amssymb, amsmath, amsfonts, amsthm, mathtools, nicefrac, bm, dsfont, upgreek}
\usepackage{mathrsfs}
\usepackage{tensor}
\usepackage{fixmath}
    \usefonttheme[onlymath]{serif}
\usepackage{physics,braket,siunitx}
\usepackage{csquotes}
\usepackage{hyperref}
    \renewcommand{\thefootnote}{\roman{footnote}}
%
\renewcommand{\vec}[1]{\bm{#1}}
\newcommand{\mat}[1]{\vphantom{\uuline{#1}}\smash{\uuline{\mkern-1mu \vphantom{\underline{#1}}#1\mkern-1mu}\mkern2mu}}
\newcommand{\uvec}[1]{\vphantom{\underline{#1}}\smash{\underline{\mkern-1mu #1\mkern-1mu}\mkern2mu}}

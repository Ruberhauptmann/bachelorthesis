\documentclass[main.tex]{subfiles}

\begin{document}
\chapter{Motivation}

For a realistic description of matter, description derived from first principle (so called ab-initio methods) are needed.
Phenomenons explainable only from ab-initio methods range from the thermodynamic properties of matter which are directly tied to phonons, i.e. quasi particles emerging in the quantization of vibrational modes, to phenomenons like superconductivity, which still don't have established theories explaining every kind of superconductor.

One such ab-initio method is \gls{dft}, the foundations of which were laid in 1964 by Hohenberg and Kohn \cite{hohenberg_inhomogeneous_1964} and in 1965 by Kohn and Sham \cite{kohn_self-consistent_1965}.
Since 1990, methods within the density functional formalism have been very successful across a number of disciplines in physics, chemistry and biology, with over \(160000\) publications on the topic between 1990 and 2015 \cite{jones_density_2015}.
The appeal of \gls{dft} methods lies in the fact that the complexity of calculations is reduced in such a way that while properties such as the full wave function cannot be computed, total energies are very reliably produced, which in turn enables calculations of lattice dynamics, thermodynamical properties of matter or chemical reactions.
These calculations are computationally cheap in comparison to methods working with full wave functions, so that simple systems can be simulated on a home computer today.
In software suites such as \QE \cite{giannozzi_quantum_2009,giannozzi_advanced_2017}, \gls{dft} methods are easily available today.

Going beyond simple calculations of a few atoms and towards current research questions makes parallel calculations over multiple nodes on compute cluster with hundreds or thousands of CPUs the only feasible possibility.
An important step is as such making sure the process of scaling the work across multiple processors is done in an effective manner to utilize available computing resources as efficient as possible.

Thus, the task of this thesis was two-fold:
First, examining the way \QE calculations are best parallelized on the PHYSnet cluster and then using this knowledge to run calculations for a system of current interest and possibly explain recent experimental data of this system \cite{hall_environmental_2019}.

The examined system is \TaS, a \acrfull{tmdc}.
Recent discovery of freestanding monolayers of \acrshort{tmdc}s \cite{novoselov_two-dimensional_2005} has brought these materials into focus.
\TaS in particular is notable 
\todo{a little bit more about TaS2, why its interesting}

The structure of this thesis is as follows:
first, all relevant theory needed to understand the calculations made with \QE will be outlined in ch. \ref{chap:many-body-physics}.
Following that, details regarding the computational work done, such as the concrete metrics evaluating performance as well as a description of the parallelization parameters offered by \QE will be presented in ch. \ref{ch:computation}.
Ch. \ref{ch:optimisation_scf} examines scalability of the \texttt{PWscf} module, which enables electronic structure calculations, the same is done in ch. \ref{ch:optimization_ph} for the \texttt{PHonon} module, which is used for calculation of phonon and phonon related properties. 
The results from these chapters are then used to run an optimized phonon calculation on \TaS in the charge density wave phase.
This optimized phonon calculation is then the foundation for a possible explanation of experimental data on \TaS in ch. \ref{ch:sts_gap_tas2}.

\end{document}

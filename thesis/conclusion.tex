\documentclass[main.tex]{subfiles}

\begin{document}
\chapter{Conclusion}

The benchmarks on \QE's \texttt{PWscf} and \texttt{PHonon} modules carried out in this thesis show how system size, compilers and parallelization parameters interplay to influence scaling of calculation across multiple processors.
\TaS, the system with longer times per iteration has a different scaling behavior to silicon, the system with shorter times per iteration:
the former shows linear speedup for significantly more processors than the latter.
Using the \gls{oneapi} instead of the \gls{openmpi} module leads to different scaling behavior on silicon, while minimal runtime stays the same.
This is due to the slightly longer single core performance with \gls{oneapi} carrying over into calculation of the speedup.
In calculations on \TaS, using \gls{oneapi} compilers leads to the calculations scaling well not just on a single node, but on almost two full nodes.

The different ranges for the optimal number of processors carries over to the choice of parameters for \QE's \(k\)-point parallelization.
For calculations on both systems, the optimal size of processor pools for \(k\)-point parallelization is near the number of processors where the speedup does not follow ideal scaling anymore.
For the system sizes at hand, utilizing linear-algebra parallelization has no impact in most cases and slows down the calculation in the case of electronic structure calculations on silicon.
In phonon calculations, using a combination of \(k\)-point and image parallelization leads to the calculations scaling over a wide range of processors.

The calculated phonon modes and frequencies on \TaS in the charge-density-wave phase give a possible explanation for a gap around the Fermi level observed in an \acrshort{sts} experiment on the material.
This gap forms via an inelastic tunneling process involving the amplitude mode near the \(\Gamma\) point opening up just for electron energies higher than that of this phonon mode.
While the energy does not fit perfectly to the size of the gap, the symmetric form of the gap would be explained.
Further theoretical work is needed to confirm this proposition.

\end{document}
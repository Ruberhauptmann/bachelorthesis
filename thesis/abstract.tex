\documentclass[main.tex]{subfiles}

\begin{document}
\section*{Kurzzusammenfassung}

Diese Arbeit untersucht \QE, eine Sammlung von Programmen für Berechnungen von elektronischen Strukturen und Modellierung von Materialien in Bezug auf seine Skalierbarkeit über mehrere Prozessoren auf dem PHYSnet compute cluster.
Die Methode ist eine Reihe an Benchmarks zum Testen von verschiedenen Compiler Kombinationen und den Parallelisierungsoptionen von \QE.
Diese Benchmarks zeigen, dass die Nutzung von Compilern in \gls{oneapi} die Skalierbarkeit signifikant verbessert und dass außerdem die Nutzung der Parallelisierungsoptionen von \QE die Rechnungen weit über einen Rechnerknoten hinweg ermöglichen, wenn sie richtig genutzt werden.
Ergebnisse aus den Benchmarks wurden außerdem genutzt, um effiziente Phononen Rechnungen von \TaS in einer Ladungsdichtewellephase duchzuführen, deren Ergebnisse möglicherweise eine Lücke um das Fermi-Niveau zu erklären, die 2019 in einem \gls{sts} Experiment an diesem Material \cite{hall_environmental_2019} gefunden wurde.

\section*{Abstract}

This thesis examines \QE, a suite of computer code for electronic-structure calculations and materials modeling in terms of its scalability on multiple processors on the PHYSnet compute cluster.
A series of benchmarks is carried out to test different combination of compilers as well as parallelization parameters offered by \QE itself.
These benchmarks show that using a set of compilers and auxiliary code in \gls{oneapi} significantly improves scaling and that the parallelization parameters offered by \QE let calculations scale beyond a single node when used right.
Results from these benchmarks were then used to carry out efficient phonon calculations on \TaS in a charge density wave phase, the results of which could explain a gap feature near the Fermi niveau observed in a 2019 \gls{sts} experiment on this material \cite{hall_environmental_2019}.

\end{document}

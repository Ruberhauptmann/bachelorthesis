\documentclass[main.tex]{subfiles}

\begin{document}
\section*{Kurzzusammenfassung}

\section*{Abstract}

This thesis examines \QE, a suite of computer code for electronic-structure calculations and materials modeling in terms of its scalability on multiple processors on the PHYSnet compute cluster.
A series of benchmarks is carried out to test different combination of compilers as well as parallelization parameters offered by \QE itself.
These benchmarks show that using a set of compilers and auxiliary code in \gls{oneapi} offered by Intel significantly improves scaling and that the parallelization parameters offered by \QE let calculations scale beyond a single node when used right.
Results from these benchmarks were then used to carry out phonon calculations on \TaS in charge density wave phase, the results of which could possibly explain a gap feature near the Fermi niveau observed in a 2019 \gls{sts} experiment on this material.

\end{document}

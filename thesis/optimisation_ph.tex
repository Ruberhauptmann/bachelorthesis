\documentclass[main.tex]{subfiles}

\begin{document}
\chapter{Parallelization of DFPT calculations}\label{ch:optimization_ph}

Calculations with the \texttt{PHonon} package are significantly more time intensive than \texttt{PWscf} calculations, so good parallelization is of the essence to make these calculations manageable.

\section{Optimal parallelization parameters for DFPT calculations}

As discussed in sec. \ref{sub:qe_parallelization}, the \texttt{PHonon} package offers the same three parallelization levels as the \texttt{PWscf} package, namely plane wave, k point and linear algebra parallelization.
Furthermore parallelization across q points (so called image parallelization) can be employed, this will be discussed separately in sec. \ref{sec:scaling_ph_images}.

\subsection{k point parallelization}

In a first step, the same k point parallelization benchmark as in sec. \ref{sub:scf_scaling_k_point} is run. This is pictured in fig. \ref{fig:scaling_ph_nk_si}.

\begin{figure}[htb!]
    \centering
    \includegraphics[width=0.8\textwidth]{plots_ph/si_ph_bench_nk_speedup.pdf}
    \caption{Scalability utilizing k-point parallelization for the Si benchmarking system with three sizes of processor pools, \emph{\QE compiled with \gls{oneapi} 2021.4, \texttt{nd 1}}}
    \label{fig:scaling_ph_nk_si}
\end{figure}
Interestingly, the results from sec. \ref{sub:scf_scaling_k_point} is not reproduced here: the smallest pool size of 2 is not the one parallelizing best, but instead it is pool size 8.
Furthermore, for more than 50 processors, even the biggest pool size 18 shows better scaling than the pool size 2.
This is similar to the results in the \texttt{PWscf} benchmark with k point parallelization on the \TaS benchmarking system in sec. \ref{sub:scf_scaling_k_point}, as the separation between the different pool sizes isn't as clear as in the same benchmark on the silicon benchmarking system.
\todo{some more interpretation possible}

\begin{figure}[htb!]
    \begin{subfigure}[b]{0.49\textwidth}
        \centering
        \includegraphics[width=\textwidth]{plots_ph/si_ph_bench_nk_absolute.pdf}
    \end{subfigure}
    \begin{subfigure}[b]{0.49\textwidth}
        \centering
        \includegraphics[width=\textwidth]{plots_ph/si_ph_bench_nk_wait.pdf}
    \end{subfigure}
    \caption{Absolute runtime and wait time for the scalability test utilizing k-point parallelization for the Si benchmarking system with three sizes of processor pools, \emph{\QE compiled with \gls{oneapi} 2021.4, \texttt{nd 1}}}
    \label{fig:scaling_ph_nk_si_absolute_wait}
\end{figure}


The wait time of up to \(\SI{50}{\percent}\) of the \gls{wall_time} 

reveals that load balancing for phonon calculations is not as easily done as for the electronic structure calculations, at least with just k point parallelization used.



\subsection{Linear algebra parallelization}

\section{Image parallelization}\label{sec:scaling_ph_images}

\todo{Better introduction}
When using image parallelization, \QE outputs a separate time report for every image, so one step is added to the analysis:
The total runtime of a calculation is determined by the longest running image, so speedup will be calculated using that value, but another important measure to evaluate is variation of times between images.
This is pictured in fig. \ref{fig:scaling_ph_ni_poolsize_8_si}.

\begin{figure}[ht!]
    \centering
    \includegraphics[width=0.8\textwidth]{plots_ph/si_ph_poolsize_8_images_distribution.pdf}
    \caption{Average runtime across images for the scalability test utilizing image and k point parallelization on the Si benchmarking system with three values of \emph{\texttt{ni}}, \emph{\QE compiled with \gls{oneapi} 2021.4, \texttt{nk, ni} chosen such that poolsize = 8, \texttt{nd 1}}}
    \label{fig:scaling_ph_ni_poolsize_8_si_distribution}
\end{figure}
As the times between images don't vary much, good load balancing between images can be assumed for the silicon benchmarking system.

With the maximum time across images, speedup is then calculated, pictured in fig. \ref{fig:scaling_ph_ni_poolsize_8_si}.

\begin{figure}[ht!]
    \centering
    \includegraphics[width=0.8\textwidth]{plots_ph/si_ph_poolsize_8_bench_ni_speedup.pdf}
    \caption{Speedup calculated from the longest running image for the scalability test utilizing image and k point parallelization on the Si benchmarking system with three values of \emph{\texttt{ni}}, \emph{\QE compiled with \gls{oneapi} 2021.4, \texttt{nk, ni} chosen such that poolsize = 8, \texttt{nd 1}}}
    \label{fig:scaling_ph_ni_poolsize_8_si}
\end{figure}


\begin{figure}[ht!]
    \centering
    \includegraphics[width=0.8\textwidth]{plots_ph/si_ph_poolsize_8_bench_ni_wait.pdf}
    \caption{Wait time calculated from the longest running image for the scalability test utilizing image and k point parallelization on the Si benchmarking system with three values of \emph{\texttt{ni}}, \emph{\QE compiled with \gls{oneapi} 2021.4, \texttt{nk, ni} chosen such that poolsize = 8, \texttt{nd 1}}}
    \label{fig:scaling_ph_ni_poolsize_8_si_wait}
\end{figure}

\section{Phonon calculations on \TaS}

The results from the last section can be used to estimate good parallelization parameters for a phonon calculation at the \(\mathrm{\Gamma}\) point for \TaS in the charge density wave phase.
The calculations were run on 180 processors, once with the previous established optimal pool size of 36 and once with a pool size of 18 for comparison.
The relevant benchmark values for this calculation are listed in tab. \ref{tab:tas2_cdw_phonon_times}.

\begin{table}[ht!]
    \caption{CAPTION}
    \begin{tabular}{@{}lll@{}}
    \toprule
                 & runtime            & wait time \\ \midrule
    pool size 18 & \SI{3044}{\minute} & 0.16         \\
    pool size 36 & \SI{2020}{\minute} & 0.074
    \end{tabular}
    \label{tab:tas2_cdw_phonon_times}
\end{table}
In this calculation the need for a good choice of parallelization parameters becomes especially clear:
on the on the same number of processors, with the only difference in the choice of the parameter \texttt{nk}, the two calculations have a difference of \(\SI{17}{\hour}\).

\section{Conclusion: Parameters for optimal scaling}

\end{document}
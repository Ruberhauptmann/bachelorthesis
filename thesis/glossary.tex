%
% Acronyms
\newacronym{dft}{DFT}{Density Functional Theory}

\newacronym{kohn_sham}{KS}{Kohn-Sham (equations)}

\newacronym{pp}{PP}{Pseudopotentials}

% 
% Glossary
\newglossaryentry{openmpi}{
    name={OpenMPI},
    description={Open source \gls{mpi} implementation}}

\newglossaryentry{mpi}{
    name={MPI},
    description={(Message Passing Interface) communication protocol for programming parallel computers}}

\newglossaryentry{openblas}{
    name={OpenBLAS},
    description={Open source implementation of the \gls{blas} and \gls{lapack} APIs}}

\newglossaryentry{oneapi}{
    name={Intel oneAPI},
    description={Intel implementation of the oneAPI specification, providing among others \gls{mpi} with C/Fortran compilers, implementations of the \gls{blas} and \gls{lapack}/\gls{scalapack} APIs all optimized for Intel processors}}

\newglossaryentry{lapack}{
    name={LAPACK},
    description={(Linear Algebra Package) software package for solving systems of simultaneous linear equations, least-squares solutions of linear systems of equations, eigenvalue problems, and singular value problem, using the \gls{blas} routines}}

\newglossaryentry{blas}{
    name={BLAS},
    description={(Basic Linear Algebra Subprograms) specification for routines that provide standard building blocks for performing basic vector and matrix operations}}

\newglossaryentry{fft}{
    name={FFT},
    description={(Fast Fourier Transform) Algorithm computing discrete fourier transforms}}

\newglossaryentry{scalapack}{
    name={ScaLAPACK},
    description={(Scalable \gls{lapack}) Implementation of a subset of \gls{lapack} routines intended to use the advantages of running on parallel machines}}

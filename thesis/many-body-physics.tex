\documentclass[main.tex]{subfiles}

\begin{document}
\chapter{Many-body physics\label{chap:many-body-physics}}
%\epigraph{It's a secret to everybody.}{a Moblin in \textit{The Legend of Zelda}}

\section{The electronic structure problem\label{sec:theory_schrödinger}}

In solid state physics, one general problem we are concerned with is finding the properties of the ground state of an isolated system of \(N\) interacting electrons in an external potential.
The system is described by the Schrödinger equation
\begin{equation}
    \ope{H} \Psi (\vb{r}_1, \ldots, \vb{r}_N) = E \Psi (\vb{r}_1, \ldots, \vb{r}_N)
\end{equation}
with the Hamiltonian in first quantization (\(i\) running over the electrons, \(\alpha, \beta\) over the nuclei)
\begin{align}
    \ope{H} &= \ope{T}_e + \ope{V}_{n-e} + \ope{V}_{e-e} + \ope{V}_{n-n} \\
    &= -\sum_i \frac{1}{2} \nabla^2_i - \sum_i \sum_{\alpha} \frac{Z_{\alpha}}{\vert \vb{r}_i - \vb{R}_{\alpha} \vert} + \frac{1}{2} \sum_{i \neq j} \frac{1}{\vert \vb{r}_i - \vb{r}_j \vert} + \frac{1}{2} \sum_{\alpha \beta} \frac{Z_{\alpha} Z_{\beta}}{R_{\alpha \beta}}
\end{align}
where:
\begin{itemize}
    \item \(\ope{T}_e\) is the kinetic energy of the electrons
    \item \(\ope{V}_{n-e}\) is the potential energy of the electrons in the field of the nuclei
    \item \(\ope{U}_{e-e}\) is the Coulomb interaction between the electrons and
    \item \(\ope{W}_{n-n}\) is the Coulomb interaction between the nuclei
\end{itemize}

\section{Density Functional Theory\label{sec:theory_dft}}

A direct solution to the electronic structure problem, this meaning obtaining the ground-state many-body wavefunction \(\Psi (\vb{r_1}, \ldots, \vb{r}_N)\) for a given potential is analytically impossible even for a small number of electrons compared to the number of electrons in a macroscopic crystal.
As such, the need for good approximations to obtain results for real world
systems is high.
One particularly successful approach is \acrfull{dft}.
In the following section, the theoretical framework of \acrshort{dft} will be developed, following \cite{marzari_ab-initio_1996}.

\subsection{Hohenberg-Kohn theorems}

The basis for DFT lies in the exact reformulation of the outlined electronic structure problem by Hohenberg and Kohn \cite{hohenberg_inhomogeneous_1964}.
This reformulation uses the ground state density of the electronic system as the basic variable. 
\todo{stuff missing}

Hohenberg-Kohn theorems:

\begin{enumerate}[I]
    \item The external potential (and by extension the ground state wave function and the ground state energy) are  unique functionals of the ground state density (except for an additive constant).
    \item The ground state energy minimizes the energy functional,
    \[E[n(r)] > E_0 \;\forall n(r) \neq n_0 (r)\].
\end{enumerate}

\subsection{Kohn-Sham equations}

One way of approximating the functional \(F[n]\) was given by Kohn and Sham \cite{kohn_self-consistent_1965}.
The idea is to use a non-interacting auxiliary system of electrons 
\begin{equation}
    H_0 = \sum_i^{N_e} \frac{p_i^2}{2m} + v_{KS} (\vb{r}_i)
\end{equation}
With a correction potential \(v_{KS}\) such that the ground state charge density for the auxiliary and the interacting system are the same.
This introduces a new set of orthonormal wave functions, the solutions to the non-interacting problem \(\Psi_i\).
This gives rise to a kinetic energy 

\subsection{Pseudopotentials}

\section{Density Functional Perturbation Theory}

\end{document}

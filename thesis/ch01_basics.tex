% command for path of images (needs to be different in subfiles)
\providecommand{\main}{.}
\documentclass[\main/main.tex]{subfiles}

\begin{document}
\chapter{Physical and Computational Basics\label{chap:theory}}
%\epigraph{It's a secret to everybody.}{a Moblin in \textit{The Legend of Zelda}}

\section{The Schrödinger Equation\label{sec:theory_schrödinger}}

\begin{equation}
    \ope{H} \Psi (\vb{r_1}, \ldots, \vb{r_N}) = E \Psi (\vb{r_1}, \ldots, \vb{r_N})
\end{equation}

The Hamiltonian for an isolated system of \(N\) interacting electrons in an external potential, so the 

\section{Density Functional Theory\label{sec:theory_dft}}

A direct solution to the electronic structure problem, this meaning obtaining the ground-state many-body wavefunction \(\Psi (\vb{r_1}, \ldots, \vb{r_N})\) for
a given potential is analytically impossible even for a small number of electrons. As such, the need for good approximations to obtain results for real world
systems is high. One particularly successful approach is \emph{Density Functional Theory} (DFT).

%Describing the approach of DFT comes in two parts: first, the exact reformulation of the electronic structure problem in terms of the electronic ground state
%density and second the approximation made by 

\subsection{Hohenberg-Kohn theorems}

\subsection{Kohn-Sham equations}

The success of DFT


\end{document}
